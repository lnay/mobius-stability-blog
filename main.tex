\documentclass[]{article}
\usepackage{projpreamble}
\newcommand{\RR}{\mathbb{R}}
\newcommand{\ZZ}{\mathbb{Z}}
\newcommand{\CC}{\mathbb{C}}
\newcommand{\triangcat}{\mathcal{T}}
\newcommand{\aut}{\mathrm{Aut}}
\newcommand{\stab}{\mathrm{Stab}}
\newcommand{\PSL}{\mathrm{PSL}}

\newcommand{\gltworplustilde}{\widetilde{\mathrm{GL}}(2,\RR)}


\begin{document}
\Configure{AddCss}{custom.css}
\Css{
	div.newtheorem {
			border-left: red 0.3em solid;
			padding: 1em 0 1em 1em;
			background-color: rgba(255, 0, 0, 0.1);
			border-radius: 0 1em 0 0;
		}
	h1, h2, h3, h4, h5, h6 {
			font-family: "Libertinus Sans";
		}
	p {
			font-family: "Libertinus Serif";
		}
	math {
			font-family: "Libertinus Math";
		}
}
\maketitle
\tableofcontents

\section{Heading}
aoeuoaeu

\section{Main statements}

\begin{theorem}
	There is a group action:
	\[
		\aut\left(\triangcat\right)^{\mathrm{num}}
		\circlearrowright
		\frac{
			\stab^{\mathrm{num}}(\triangcat)
		}{
			\gltworplustilde^+
		}.
	\]
	\noindent
	In the case of a principally polarised abelian surface, descends to:
	\[
		\PSL(2, \ZZ)
		\circlearrowright
		\left\{
		z \in \CC
		\colon
		\mathfrak{Im}(z) > 0
		\right\},
	\]
	coinciding with the standard action of
	the ``modular group'' $\PSL(2, \ZZ)$
	on the upper half-plane via Möbius transformations.
\end{theorem}

\end{document}
